% Options for packages loaded elsewhere
\PassOptionsToPackage{unicode}{hyperref}
\PassOptionsToPackage{hyphens}{url}
%
\documentclass[
]{article}
\usepackage{amsmath,amssymb}
\usepackage{lmodern}
\usepackage{iftex}
\ifPDFTeX
  \usepackage[T1]{fontenc}
  \usepackage[utf8]{inputenc}
  \usepackage{textcomp} % provide euro and other symbols
\else % if luatex or xetex
  \usepackage{unicode-math}
  \defaultfontfeatures{Scale=MatchLowercase}
  \defaultfontfeatures[\rmfamily]{Ligatures=TeX,Scale=1}
\fi
% Use upquote if available, for straight quotes in verbatim environments
\IfFileExists{upquote.sty}{\usepackage{upquote}}{}
\IfFileExists{microtype.sty}{% use microtype if available
  \usepackage[]{microtype}
  \UseMicrotypeSet[protrusion]{basicmath} % disable protrusion for tt fonts
}{}
\makeatletter
\@ifundefined{KOMAClassName}{% if non-KOMA class
  \IfFileExists{parskip.sty}{%
    \usepackage{parskip}
  }{% else
    \setlength{\parindent}{0pt}
    \setlength{\parskip}{6pt plus 2pt minus 1pt}}
}{% if KOMA class
  \KOMAoptions{parskip=half}}
\makeatother
\usepackage{xcolor}
\IfFileExists{xurl.sty}{\usepackage{xurl}}{} % add URL line breaks if available
\IfFileExists{bookmark.sty}{\usepackage{bookmark}}{\usepackage{hyperref}}
\hypersetup{
  hidelinks,
  pdfcreator={LaTeX via pandoc}}
\urlstyle{same} % disable monospaced font for URLs
\setlength{\emergencystretch}{3em} % prevent overfull lines
\providecommand{\tightlist}{%
  \setlength{\itemsep}{0pt}\setlength{\parskip}{0pt}}
\setcounter{secnumdepth}{-\maxdimen} % remove section numbering
\ifLuaTeX
  \usepackage{selnolig}  % disable illegal ligatures
\fi

\author{}
\date{}

\begin{document}

RELATÓRIO COMPLETO -- Sistema HospitalDB

Este relatório apresenta, descreve e justifica todas as etapas
envolvidas no desenvolvimento do projeto HospitalDB, criado com o
objetivo de fornecer uma solução organizada e funcional para o
gerenciamento básico de informações dentro de um ambiente hospitalar. O
sistema foi desenvolvido em três partes principais: modelagem do banco
de dados, implementação física em SQL e criação de uma aplicação Python
para interação com o banco de dados. A seguir, são detalhadas todas as
decisões técnicas e operacionais adotadas ao longo do desenvolvimento do
projeto.

\begin{enumerate}
\def\labelenumi{\arabic{enumi}.}
\tightlist
\item
  Criação do Banco de Dados
\end{enumerate}

O projeto teve início com a criação de um novo banco denominado
HospitalDB, utilizando o mecanismo InnoDB. A escolha do InnoDB foi
fundamental para garantir a utilização de chaves estrangeiras, ações de
cascata e integridade referencial, fatores essenciais para a
consistência de dados em um sistema hospitalar. Antes da criação,
certificou-se de que nenhuma versão anterior do banco permanecia ativa,
permitindo a construção de um ambiente completamente limpo e organizado.

\begin{enumerate}
\def\labelenumi{\arabic{enumi}.}
\setcounter{enumi}{1}
\tightlist
\item
  Estrutura e Modelagem das Tabelas
\end{enumerate}

A modelagem foi planejada para representar de maneira simples, clara e
funcional os elementos essenciais de um hospital. As tabelas definidas
foram Paciente, Medico e Consulta, representando o núcleo básico de
qualquer sistema de atendimento.

2.1. Tabela Paciente\\
A tabela Paciente foi desenvolvida para armazenar as informações
pessoais dos indivíduos atendidos no hospital. O CPF foi adotado como
chave primária por ser um identificador único e amplamente utilizado. Os
atributos definidos foram: cpf, nome, telefone e dataNascimento. Essa
estrutura garante que o sistema só permita consultas associadas a
pacientes devidamente cadastrados.

2.2. Tabela Medico\\
A tabela Medico representa os profissionais de saúde. Foi criada com um
campo id de auto incremento como chave primária, facilitando a
organização e identificação dos médicos. Os atributos armazenados são:
id, nome e especialidade. Essa tabela se relaciona diretamente com a
tabela de consultas, permitindo saber qual médico realizou cada
atendimento.

2.3. Tabela Consulta\\
A tabela Consulta registra os atendimentos realizados no hospital. É
nela que ocorre o vínculo entre pacientes e médicos. Os atributos
definidos foram: id, dataHora, cpfPaciente e idMedico. A modelagem
utiliza duas chaves estrangeiras, vinculadas às tabelas Paciente e
Medico, ambas configuradas com ON DELETE CASCADE, garantindo que a
exclusão de um paciente ou médico remova automaticamente consultas
dependentes, evitando registros órfãos e preservando a integridade do
banco.

\begin{enumerate}
\def\labelenumi{\arabic{enumi}.}
\setcounter{enumi}{2}
\tightlist
\item
  População do Banco de Dados
\end{enumerate}

Após criar as tabelas, foi desenvolvido um script de inserção de dados
para gerar registros iniciais de testes. Foram cadastrados pacientes
(como João Silva e Maria Souza), médicos (como Dr.~Carlos Mendes e Dra.
Ana Pereira) e consultas associadas entre eles. Esses dados permitem que
a aplicação Python execute consultas reais e demonstre plenamente o
funcionamento do sistema.

\begin{enumerate}
\def\labelenumi{\arabic{enumi}.}
\setcounter{enumi}{3}
\tightlist
\item
  Desenvolvimento da Aplicação Python
\end{enumerate}

Para possibilitar a interação do usuário com o banco de dados, foi
criada uma aplicação em Python utilizando a biblioteca mysql.connector.
Essa aplicação funciona como uma camada de comunicação entre o usuário e
o banco, permitindo consultar pacientes, médicos e consultas de forma
simples e direta.

A aplicação foi estruturada em funções responsáveis por estabelecer a
conexão com o banco, executar comandos SQL e exibir resultados de
maneira clara. O menu apresentado ao usuário contém as opções:
visualizar pacientes, visualizar médicos, visualizar consultas e sair.
Cada opção envia um comando SQL ao banco e retorna os dados cadastrados,
demonstrando a integração entre Python e SQL.

\begin{enumerate}
\def\labelenumi{\arabic{enumi}.}
\setcounter{enumi}{4}
\tightlist
\item
  Conclusão
\end{enumerate}

O projeto HospitalDB apresenta um sistema funcional, organizado e
coerente, capaz de realizar o gerenciamento básico de informações
hospitalares. O banco de dados foi estruturado com foco em integridade e
clareza, utilizando chaves primárias e estrangeiras adequadamente. A
aplicação Python demonstra a conexão prática entre código e banco de
dados, permitindo que a teoria da modelagem e dos comandos SQL seja
aplicada em um ambiente real.

O sistema está preparado para futuras expansões, como inclusão de
tabelas de internações, exames, prontuários ou setores do hospital,
mantendo sua flexibilidade e escalabilidade. Dessa forma, o projeto
atende plenamente os requisitos solicitados e demonstra domínio sobre
modelagem de dados, SQL e integração com Python.

\end{document}
